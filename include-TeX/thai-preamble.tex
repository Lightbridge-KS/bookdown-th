%% ---- ตั้งค่าให้ตัดคำภาษาไทย ---- %%
\XeTeXlinebreaklocale "th"
\XeTeXlinebreakskip = 0pt plus 0pt % เพิ่มความกว้างเว้นวรรคให้ความยาวแต่ละบรรทัดเท่ากัน


%% ---- font settings ---- %%
\usepackage{fontspec} % For Thai font
\defaultfontfeatures{Mapping=tex-text} % map LaTeX formating, e.g., ``'', to match the current font
% To change the main font, uncomment one of the below command.
% \setmainfont{TeX Gyre Termes} % Free Times
% \setsansfont{TeX Gyre Heros} % Free Helvetica
% \setmonofont{TeX Gyre Cursor} % Free Courier
\newfontfamily{\thaifont}[Scale=MatchUppercase,Mapping=textext]{TH Sarabun New} % ตั้งฟอนต์หลักภาษาไทย ที่น่าใช้: "Laksaman", "TH Sarabun New"
\newenvironment{thailang}{\thaifont}{} % create environment for Thai language
\usepackage[Latin,Thai]{ucharclasses} % ตั้งค่าให้ใช้ "thailang" environment เฉพาะ string ที่เป็น Unicode ภาษาไทย

\setTransitionTo{Thai}{\begin{thailang}}
\setTransitionFrom{Thai}{\end{thailang}}

%% ---- spacing between lines ---- %%
\usepackage{setspace}
% \singlespacing % default setting
% recommend using one-half spacing for Thai language
\onehalfspacing % Set globally here or using as an environment

%% ---- using alphabatic language ---- %%
\usepackage{polyglossia}
\setdefaultlanguage{english} % it is preferrable to set English as the main language, since the numeric system is compatible with most LaTeX features such as 'enumerate' and so on
\setotherlanguages{thai}

\AtBeginDocument\captionsthai % allow captions to be in Thai

%% ---- hyperref settings ---- %%
\usepackage{hyperref}
\usepackage{url}
\usepackage{cite}
\usepackage{xcolor}
\hypersetup{
    colorlinks,
    linkcolor={red!50!black},
    citecolor={blue!50!black},
    urlcolor={blue!80!black}
    }


